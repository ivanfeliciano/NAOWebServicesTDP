
Esta interfaz es el producto de la integración de modelos de aprendizaje
automático
enfocados a casos de uso relacionados con la robótica.
% Es uno de los componentes más importantes en toda la arquitectura,
% es el que une los servicios web de
% terceros, los módulos mantenidos por el LAR y los entrega en una sola API.

La API REST  permite integrar dentro de una aplicación un
conjunto de herramientas para el análisis de imágenes.
Todas éstas basadas en modelos de aprendizaje automático. El URL base al
que todos los recursos son relativos es \sphinxcode{http://132.248.180.17/}.

La API tiene tres recursos:
\begin{itemize}
\item {} 
\sphinxcode{/register}: para que un nuevo usuario se registre y pueda ocupar los otros recursos. Envía una dirección de correo y una contraseña.

\item {} 
\sphinxcode{/refreshtoken}: para que el usuario pueda obtener un nuevo token de acceso al enviar los datos con los que se registró.

\item {} 
\sphinxcode{/vision}: el recurso más importante. Se le solicita que haga procesamiento de imágenes.

\end{itemize}

\subsection{Vision}
\label{\detokenize{chapter_two/desc_cloudnao:vision}}
Este recurso es el encargado de la detección de características en una
imagen. Estas características son: la \sphinxstylestrong{detección de objetos}, el
\sphinxstylestrong{reconocimiento de rostros}, de personas previamente guardadas o de
nuevos sujetos para su almacenamiento, la \sphinxstylestrong{clasificación en cuatro escenarios}, lugares
dentro del laboratorio de algoritmos para la robótica, la
\sphinxstylestrong{detección de etiquetas o categorías}
y la \sphinxstylestrong{traducción de texto encontrado en una imagen}.

En este recurso la única forma de persistencia de datos es al guardar el
rostro de una nueva persona para su posterior reconocimiento.


\subsubsection{POST}
\label{\detokenize{chapter_two/desc_cloudnao:post-vision}}

Se envía una imagen ya sea codificada en base 64 o mediante su URL y
las características que se desean obtener. Las características disponibles
son las siguientes:

\begin{itemize}
    \item{}
    \sphinxcode{FACE\_ENROLL}:
Detecta un rostro en la imagen y
con un identificador enviado en
el cuerpo de la petición se
almacena en una galería de
Kairos. Se emplea
\texttt{/enroll}
de Kairos.
\item{}
\sphinxcode{FACE\_RECOGNITION}:
Encuentra rostros dentro de una
imagen y los relaciona con los
rostros similares previamente
guardados en una galería de
Kairos. Usa
\texttt{/recognize}
de la API de Kairos.
\item{}
\sphinxcode{OBJECT\_DETECTION}:
Busca objetos en una la imagen,
usa la API de detecctión de objetos de TensorFlow.
Regresa las coordenadas de un
cuadro delimitador para cada
objeto detectado.
\item{}
\sphinxcode{LABELS\_DETECTION}:
Clasifica una imagen en distintas
categorías. Se vale de la API de Google Cloud Vision
\item{}
\sphinxcode{OCR\_TRANSLATION}:
Lleva a cabo el reconocimiento de
texto en una imagen, para
posteriormente traducirlo.
Utiliza la API de Google Cloud
Vision
para el reconocimiento de
caracteres y la de Google Cloud
Translation
para la segunda parte del proceso.
\item{}
\sphinxcode{CLASSIFY\_INDOOR\_SCENES}:
Clasifica una imagen en cuatro
categorías. Cada categoría es un
lugar dentro del área del LAR.

\end{itemize}


\paragraph{Mensaje de solicitud HTTP}
\label{\detokenize{chapter_two/desc_cloudnao:peticion}}

\subparagraph{Headers.}
\label{\detokenize{chapter_two/desc_cloudnao:headers}}
En los encabezados de la petición debe de ir el tipo de contenido que se envía
y un token de acceso único para cada usuario registrado. Esto último
simplemente es para evitar que cualquiera pueda hacer peticiones a la API.

\begin{sphinxVerbatim}[commandchars=\\\{\}]
\PYG{n}{Content}\PYG{o}{\PYGZhy{}}\PYG{n}{Type}\PYG{p}{:} \PYG{n}{application}\PYG{o}{/}\PYG{n}{json}
\PYG{n}{Authorization}\PYG{p}{:} \PYG{n}{ACCESS\PYGZus{}TOKEN}
\end{sphinxVerbatim}


\subparagraph{\sphinxstylestrong{Cuerpo.}}
\label{\detokenize{chapter_two/desc_cloudnao:body}}
En el cuerpo del mensaje se envía un JSON con la siguiente estructura. En la tabla 
\ref{tab:body_json_schema_request} se describen los atributos del JSON.

\begin{sphinxVerbatim}[commandchars=\\\{\}]
\PYG{p}{\PYGZob{}}
  \PYG{l+s+s2}{\PYGZdq{}imageContent\PYGZdq{}}\PYG{o}{:} \PYG{l+s+s2}{\PYGZdq{}Hello, world!\PYGZdq{}}\PYG{p}{,}
  \PYG{l+s+s2}{\PYGZdq{}imageSource\PYGZdq{}}\PYG{o}{:} \PYG{l+s+s2}{\PYGZdq{}Hello, world!\PYGZdq{}}\PYG{p}{,}
  \PYG{l+s+s2}{\PYGZdq{}features\PYGZdq{}}\PYG{o}{:} \PYG{p}{[}
    \PYG{p}{\PYGZob{}}
      \PYG{l+s+s2}{\PYGZdq{}type\PYGZdq{}}\PYG{o}{:} \PYG{l+s+s2}{\PYGZdq{}Hello, world!\PYGZdq{}}\PYG{p}{,}
      \PYG{l+s+s2}{\PYGZdq{}subjectID\PYGZdq{}}\PYG{o}{:} \PYG{l+s+s2}{\PYGZdq{}Hello, world!\PYGZdq{}}
    \PYG{p}{\PYGZcb{}}
  \PYG{p}{]}
\PYG{p}{\PYGZcb{}}
\end{sphinxVerbatim}


\begin{table}[ht]
\caption{Descripción de los elementos del JSON del cuerpo de la solicitud.\label{tab:body_json_schema_request}}
\begin{tabular}{|l|l}
\hline
\multicolumn{2}{|l|}{\cellcolor[HTML]{68CBD0}\textbf{Propiedades}}                                                                                                                                                                                                      \\ \hline
\multicolumn{2}{|l|}{\cellcolor[HTML]{68CBD0}\texttt{imageContent}}                                                                                                                                                                                                       \\ \hline
Tipo de dato & \multicolumn{1}{l|}{string}                                                                                                                                                                                                      \\ \hline
Descripción  & \multicolumn{1}{l|}{Imagen codificada en base 64.}                                                                                                                                                                               \\ \hline
\multicolumn{2}{|l|}{\cellcolor[HTML]{68CBD0}\texttt{imageSource}}                                                                                                                                                                                                       \\ \hline
Tipo de dato & \multicolumn{1}{l|}{string}                                                                                                                                                                                                      \\ \hline
Descripción  & \multicolumn{1}{l|}{URL público de la imagen.}                                                                                                                                                                                   \\ \hline
\multicolumn{2}{|l|}{\texttt{features}}                                                                                                                                                                                      \\ \hline
Tipo de dato & \multicolumn{1}{l|}{array}                                                                                                                                                                                                       \\ \hline
\multirow{4}{*}{Descripción}  & 
\multicolumn{1}{l|}{
Una arreglo de las características que se desean detectar
en la} \\
&
\multicolumn{1}{l|}{imagen. Se debe solicitar al menos una de las seis disponibles.} \\&
\multicolumn{1}{l|}{Por ejemplo FACE\_ENROLL, FACE\_RECOGNITION,} \\& 
\multicolumn{1}{l|}{CLASSIFY\_INDOOR\_SCENES, etc.}


\\ \hline
\multicolumn{2}{|l|}{\cellcolor[HTML]{68CBD0}\textbf{Campos obligatorios}}                                                                                                                                                                                              \\ \hline
\multicolumn{2}{|l|}{imageContent}                                                                                                                                                                                                              \\
\multicolumn{2}{|l|}{imageSource}                                                                                                                                                                                                               \\
\multicolumn{2}{|l|}{features}                                                                                                                                                                                                                  \\ \hline
\end{tabular}
\end{table}



% \begin{sphinxVerbatim}[commandchars=\\\{\}]
% \PYG{p}{\PYGZob{}}
%   \PYG{l+s+s2}{\PYGZdq{}\PYGZdl{}schema\PYGZdq{}}\PYG{o}{:} \PYG{l+s+s2}{\PYGZdq{}http://json\PYGZhy{}schema.org/draft\PYGZhy{}04/schema\PYGZsh{}\PYGZdq{}}\PYG{p}{,}
%   \PYG{l+s+s2}{\PYGZdq{}type\PYGZdq{}}\PYG{o}{:} \PYG{l+s+s2}{\PYGZdq{}object\PYGZdq{}}\PYG{p}{,}
%   \PYG{l+s+s2}{\PYGZdq{}properties\PYGZdq{}}\PYG{o}{:} \PYG{p}{\PYGZob{}}
%     \PYG{l+s+s2}{\PYGZdq{}imageContent\PYGZdq{}}\PYG{o}{:} \PYG{p}{\PYGZob{}}
%       \PYG{l+s+s2}{\PYGZdq{}type\PYGZdq{}}\PYG{o}{:} \PYG{l+s+s2}{\PYGZdq{}string\PYGZdq{}}\PYG{p}{,}
%       \PYG{l+s+s2}{\PYGZdq{}description\PYGZdq{}}\PYG{o}{:} \PYG{l+s+s2}{\PYGZdq{}Imagen codificada en base 64\PYGZdq{}}
%     \PYG{p}{\PYGZcb{}}\PYG{p}{,}
%     \PYG{l+s+s2}{\PYGZdq{}imageSource\PYGZdq{}}\PYG{o}{:} \PYG{p}{\PYGZob{}}
%       \PYG{l+s+s2}{\PYGZdq{}type\PYGZdq{}}\PYG{o}{:} \PYG{l+s+s2}{\PYGZdq{}string\PYGZdq{}}\PYG{p}{,}
%       \PYG{l+s+s2}{\PYGZdq{}description\PYGZdq{}}\PYG{o}{:} \PYG{l+s+s2}{\PYGZdq{}URL de la imagen\PYGZdq{}}
%     \PYG{p}{\PYGZcb{}}\PYG{p}{,}
%     \PYG{l+s+s2}{\PYGZdq{}features\PYGZdq{}}\PYG{o}{:} \PYG{p}{\PYGZob{}}
%       \PYG{l+s+s2}{\PYGZdq{}type\PYGZdq{}}\PYG{o}{:} \PYG{l+s+s2}{\PYGZdq{}array\PYGZdq{}}\PYG{p}{,}
%       \PYG{l+s+s2}{\PYGZdq{}description\PYGZdq{}}\PYG{o}{:} \PYG{l+s+s2}{\PYGZdq{}Una arreglo de las características que se desean detectar en la imagen. Se debe solicitar al menos una de las seis disponibles. Por ejemplo FACE\PYGZus{}ENROLL, FACE\PYGZus{}RECOGNITION, CLASSIFY\PYGZus{}INDOOR\PYGZus{}SCENES, etc.\PYGZdq{}}
%     \PYG{p}{\PYGZcb{}}
%   \PYG{p}{\PYGZcb{}}\PYG{p}{,}
%   \PYG{l+s+s2}{\PYGZdq{}required\PYGZdq{}}\PYG{o}{:} \PYG{p}{[}
%     \PYG{l+s+s2}{\PYGZdq{}imageContent\PYGZdq{}}\PYG{p}{,}
%     \PYG{l+s+s2}{\PYGZdq{}imageSource\PYGZdq{}}\PYG{p}{,}
%     \PYG{l+s+s2}{\PYGZdq{}features\PYGZdq{}}
%   \PYG{p}{]}
% \PYG{p}{\PYGZcb{}}
% \end{sphinxVerbatim}


\paragraph{Mensaje de respuesta HTTP}
\label{\detokenize{chapter_two/desc_cloudnao:respuesta}}

\subparagraph{\sphinxstylestrong{Cuerpo.}}
\label{\detokenize{chapter_two/desc_cloudnao:id1}}
En el cuerpo del mensaje de respuesta, si se obtiene un código 200,
va un JSON con las características encontradas. En éste también
se incluyen mensajes de error que no pertenecen al estándar del protocolo HTTP.
En la tabla \ref{tab:json_schema_response} se describen los atributos del JSON.

\begin{sphinxVerbatim}[commandchars=\\\{\}]
\PYG{p}{\PYGZob{}}
  \PYG{l+s+s2}{\PYGZdq{}features\PYGZdq{}}\PYG{o}{:} \PYG{p}{\PYGZob{}}
    \PYG{l+s+s2}{\PYGZdq{}faceRecognition\PYGZdq{}}\PYG{o}{:} \PYG{p}{[}
      \PYG{p}{\PYGZob{}}
        \PYG{l+s+s2}{\PYGZdq{}topLeftX\PYGZdq{}}\PYG{o}{:} \PYG{l+m+mi}{1}\PYG{p}{,}
        \PYG{l+s+s2}{\PYGZdq{}topLeftY\PYGZdq{}}\PYG{o}{:} \PYG{l+m+mi}{1}\PYG{p}{,}
        \PYG{l+s+s2}{\PYGZdq{}width\PYGZdq{}}\PYG{o}{:} \PYG{l+m+mi}{1}\PYG{p}{,}
        \PYG{l+s+s2}{\PYGZdq{}height\PYGZdq{}}\PYG{o}{:} \PYG{l+m+mi}{1}\PYG{p}{,}
        \PYG{l+s+s2}{\PYGZdq{}subjectId\PYGZdq{}}\PYG{o}{:} \PYG{l+s+s2}{\PYGZdq{}Hello, world!\PYGZdq{}}\PYG{p}{,}
        \PYG{l+s+s2}{\PYGZdq{}confidence\PYGZdq{}}\PYG{o}{:} \PYG{l+m+mi}{1}
      \PYG{p}{\PYGZcb{}}
    \PYG{p}{]}\PYG{p}{,}
    \PYG{l+s+s2}{\PYGZdq{}objectDetection\PYGZdq{}}\PYG{o}{:} \PYG{p}{[}
      \PYG{p}{\PYGZob{}}
        \PYG{l+s+s2}{\PYGZdq{}category\PYGZdq{}}\PYG{o}{:} \PYG{l+s+s2}{\PYGZdq{}Hello, world!\PYGZdq{}}\PYG{p}{,}
        \PYG{l+s+s2}{\PYGZdq{}confidence\PYGZdq{}}\PYG{o}{:} \PYG{l+m+mi}{1}\PYG{p}{,}
        \PYG{l+s+s2}{\PYGZdq{}topLeftX\PYGZdq{}}\PYG{o}{:} \PYG{l+m+mi}{1}\PYG{p}{,}
        \PYG{l+s+s2}{\PYGZdq{}topLeftY\PYGZdq{}}\PYG{o}{:} \PYG{l+m+mi}{1}\PYG{p}{,}
        \PYG{l+s+s2}{\PYGZdq{}width\PYGZdq{}}\PYG{o}{:} \PYG{l+m+mi}{1}\PYG{p}{,}
        \PYG{l+s+s2}{\PYGZdq{}height\PYGZdq{}}\PYG{o}{:} \PYG{l+m+mi}{1}
      \PYG{p}{\PYGZcb{}}
    \PYG{p}{]}\PYG{p}{,}
    \PYG{l+s+s2}{\PYGZdq{}labelsDetection\PYGZdq{}}\PYG{o}{:} \PYG{p}{[}
      \PYG{p}{\PYGZob{}}
        \PYG{l+s+s2}{\PYGZdq{}name\PYGZdq{}}\PYG{o}{:} \PYG{l+s+s2}{\PYGZdq{}Hello, world!\PYGZdq{}}\PYG{p}{,}
        \PYG{l+s+s2}{\PYGZdq{}confidence\PYGZdq{}}\PYG{o}{:} \PYG{l+m+mi}{1}
      \PYG{p}{\PYGZcb{}}
    \PYG{p}{]}\PYG{p}{,}
    \PYG{l+s+s2}{\PYGZdq{}ocrTranslation\PYGZdq{}}\PYG{o}{:} \PYG{p}{\PYGZob{}}
      \PYG{l+s+s2}{\PYGZdq{}sourceText\PYGZdq{}}\PYG{o}{:} \PYG{l+s+s2}{\PYGZdq{}Hello, world!\PYGZdq{}}\PYG{p}{,}
      \PYG{l+s+s2}{\PYGZdq{}targetText\PYGZdq{}}\PYG{o}{:} \PYG{l+s+s2}{\PYGZdq{}Hello, world!\PYGZdq{}}\PYG{p}{,}
      \PYG{l+s+s2}{\PYGZdq{}sourceLanguage\PYGZdq{}}\PYG{o}{:} \PYG{l+s+s2}{\PYGZdq{}Hello, world!\PYGZdq{}}
    \PYG{p}{\PYGZcb{}}\PYG{p}{,}
    \PYG{l+s+s2}{\PYGZdq{}faceEnroll\PYGZdq{}}\PYG{o}{:} \PYG{p}{\PYGZob{}}
      \PYG{l+s+s2}{\PYGZdq{}topLeftX\PYGZdq{}}\PYG{o}{:} \PYG{l+m+mi}{1}\PYG{p}{,}
      \PYG{l+s+s2}{\PYGZdq{}topLeftY\PYGZdq{}}\PYG{o}{:} \PYG{l+m+mi}{1}\PYG{p}{,}
      \PYG{l+s+s2}{\PYGZdq{}width\PYGZdq{}}\PYG{o}{:} \PYG{l+m+mi}{1}\PYG{p}{,}
      \PYG{l+s+s2}{\PYGZdq{}height\PYGZdq{}}\PYG{o}{:} \PYG{l+m+mi}{1}\PYG{p}{,}
      \PYG{l+s+s2}{\PYGZdq{}confidence\PYGZdq{}}\PYG{o}{:} \PYG{l+m+mi}{1}\PYG{p}{,}
      \PYG{l+s+s2}{\PYGZdq{}gender\PYGZdq{}}\PYG{o}{:} \PYG{l+s+s2}{\PYGZdq{}Hello, world!\PYGZdq{}}
    \PYG{p}{\PYGZcb{}}
    \PYG{l+s+s2}{\PYGZdq{}indoorScenesClassifier\PYGZdq{}} \PYG{o}{:} \PYG{p}{\PYGZob{}}
      \PYG{l+s+s2}{\PYGZdq{}indoorScene\PYGZdq{}}\PYG{o}{:} \PYG{l+s+s2}{\PYGZdq{}Hello, world!\PYGZdq{}}
    \PYG{p}{\PYGZcb{}}
  \PYG{p}{\PYGZcb{}}\PYG{p}{\PYGZcb{}}
\end{sphinxVerbatim}

\begin{longtable}{|l|l|}
\caption{Atributos que componen el JSON del cuerpo de la respuesta.\label{tab:json_schema_response}}\\
\hline
\multicolumn{2}{|l|}{\cellcolor[HTML]{68CBD0}\texttt{features}}                                                                                                                                \\ \hline
\endfirsthead
%
\endhead
%
\cellcolor[HTML]{68CBD0}\textbf{Tipo de dato} & objeto                                                                                                                                         \\ \hline
\multicolumn{2}{|l|}{\cellcolor[HTML]{68CBD0}\textbf{Propiedades}}                                                                                                                             \\ \hline
\multicolumn{2}{|l|}{\cellcolor[HTML]{68CBD0}\texttt{faceRecognition}}                                                                                                                         \\ \hline
Tipo de dato                                  & array                                                                                                                                          \\ \hline
Descripción                                   & \begin{tabular}[c]{@{}l@{}}Un arreglo de objetos que contienen \\ a los rostros reconocidos.\end{tabular}                                      \\ \hline
\multicolumn{2}{|l|}{\cellcolor[HTML]{68CBD0}\texttt{objectDetection}}                                                                                                                         \\ \hline
Tipo de dato                                  & array                                                                                                                                          \\ \hline
Descripción                                   & Contiene un arreglo con todos los objetos detectados.                                                                                          \\ \hline
\multicolumn{2}{|l|}{\cellcolor[HTML]{68CBD0}\texttt{labelsDetection}}                                                                                                                         \\ \hline
Tipo de dato                                  & array                                                                                                                                          \\ \hline
Descripción                                   & Contiene un arreglo con las etiquetas de la imagen.                                                                                            \\ \hline
\multicolumn{2}{|l|}{\cellcolor[HTML]{68CBD0}\texttt{ocrTranslation}}                                                                                                                          \\ \hline
\cellcolor[HTML]{FFFFFF}Tipo de dato          & objeto                                                                                                                                         \\ \hline
\multicolumn{2}{|l|}{\cellcolor[HTML]{FFFFFF}Propiedades}                                                                                                                                      \\ \hline
\multicolumn{2}{|l|}{\texttt{sourceText}}                                                                                                                                                      \\ \hline
Tipo de dato                                  & string                                                                                                                                         \\ \hline
Descripción                                   & El texto en formato UTF-8.                                                                                                                     \\ \hline
\multicolumn{2}{|l|}{\texttt{targetText}}                                                                                                                                                      \\ \hline
Tipo de dato                                  & string                                                                                                                                         \\ \hline
Descripción                                   & El texto traducido.                                                                                                                            \\ \hline
\multicolumn{2}{|l|}{\texttt{sourceLanguage}}                                                                                                                                                  \\ \hline
Tipo de dato                                  & string                                                                                                                                         \\ \hline
Descripción                                   & El idioma original.                                                                                                                            \\ \hline
Descripción                                   & Un objecto con el texto encontrado en la imagen.                                                                                               \\ \hline
\multicolumn{2}{|l|}{\cellcolor[HTML]{68CBD0}\texttt{faceEnroll}}                                                                                                                              \\ \hline
Tipo de dato                                  & objeto                                                                                                                                         \\ \hline
\multicolumn{2}{|l|}{Propiedades}                                                                                                                                                              \\ \hline
\multicolumn{2}{|l|}{\texttt{topLeftX}}                                                                                                                                                        \\ \hline
Tipo de dato                                  & number                                                                                                                                         \\ \hline
Descripción                                   & Coordenada sobre el eje x.                                                                                                                     \\ \hline
\multicolumn{2}{|l|}{\texttt{topLeftY}}                                                                                                                                                        \\ \hline
Tipo de dato                                  & number                                                                                                                                         \\ \hline
Descripción                                   & Coordenada                                                                                                                                     \\ \hline
\multicolumn{2}{|l|}{\texttt{width}}                                                                                                                                                           \\ \hline
Tipo de dato                                  & number                                                                                                                                         \\ \hline
Descripción                                   & Ancho del recuadro que delimita la imagen.                                                                                                     \\ \hline
\multicolumn{2}{|l|}{\texttt{height}}                                                                                                                                                          \\ \hline
Tipo de dato                                  & number                                                                                                                                         \\ \hline
Descripción                                   & Altura del recuadro que delimita la imagen.                                                                                                    \\ \hline
\multicolumn{2}{|l|}{\texttt{confidence}}                                                                                                                                                      \\ \hline
Tipo de dato                                  & number                                                                                                                                         \\ \hline
Descripción                                   & Valor de 0-1 que representa una probabilidad.                                                                                                  \\ \hline
\multicolumn{2}{|l|}{\texttt{gender}}                                                                                                                                                          \\ \hline
Tipo de dato                                  & string                                                                                                                                         \\ \hline
Descripción                                   & Sexo de la persona con ese rostro (M o F)                                                                                                      \\ \hline
Descripción                                   & Características del rostro detectado.                                                                                                          \\ \hline
\multicolumn{2}{|l|}{\cellcolor[HTML]{68CBD0}\texttt{indoorScenesClassifier}}                                                                                                                    \\ \hline
Tipo de dato                                  & objeto                                                                                                                                         \\ \hline
\multicolumn{2}{|l|}{Propiedades}                                                                                                                                                              \\ \hline
\multicolumn{2}{|l|}{\texttt{indoorScene}}                                                                                                                                                     \\ \hline
Tipo de dato                                  & string                                                                                                                                         \\ \hline
Descripción                                   & \begin{tabular}[c]{@{}l@{}}La escena detectada, puede ser cualquiera de las\\ cuatro posibles (exit, soccer\_court, desks, office)\end{tabular} \\ \hline
Descripción                                   & Escenario reconocido.                                                                                                                          \\ \hline
\cellcolor[HTML]{68CBD0}\textbf{Descripción}  & \begin{tabular}[c]{@{}l@{}}Lista con las respuestas de acuerdo a las \\ características que se solicitaron.\end{tabular}                       \\ \hline
\end{longtable}


\subsubsection{Ejemplo de una petición a la API con cURL.}

cURL es una herramienta en la línea de comandos para transferir
datos usando diferentes protocolos. Por facilidad,
se muestra el uso de la API a través de esta herramienta.
Para solicitar el recurso \texttt{vision}, para procesamiento de imágenes,  se necesita un token de acceso. El token
se obtiene creando un usuario haciendo una petición al recurso
\texttt{register}. El fragmento de código  para solicitar este servicio con
\texttt{cURL} es el siguiente:

\fvset{hllines={, ,}}%
\begin{sphinxVerbatim}[commandchars=\\\{\}]
curl \PYGZhy{}X POST \PYG{l+s+se}{\PYGZbs{}}
http://132.248.180.17/register \PYG{l+s+se}{\PYGZbs{}}
\PYGZhy{}H \PYG{l+s+s1}{\PYGZsq{}content\PYGZhy{}type: application/json\PYGZsq{}}\PYG{l+s+se}{\PYGZbs{}}
\PYGZhy{}d \PYG{l+s+s1}{\PYGZsq{}\PYGZob{}}
\PYG{l+s+s1}{    \PYGZdq{}username\PYGZdq{} : \PYGZdq{}mock\PYGZus{}user@gmailcom\PYGZdq{},}
\PYG{l+s+s1}{    \PYGZdq{}password\PYGZdq{} : \PYGZdq{}p45sw0rd\PYGZdq{}}
\PYG{l+s+s1}{    \PYGZcb{}\PYGZsq{}}
\end{sphinxVerbatim}

La respuesta de la API es un JSON como el que sigue:

\fvset{hllines={, ,}}%
\begin{sphinxVerbatim}[commandchars=\\\{\}]
\PYG{p}{\PYGZob{}}
    \PYG{l+s+s2}{\PYGZdq{}}\PYG{l+s+s2}{message}\PYG{l+s+s2}{\PYGZdq{}}\PYG{p}{:} \PYG{l+s+s2}{\PYGZdq{}}\PYG{l+s+s2}{User created successfully.}\PYG{l+s+s2}{\PYGZdq{}}\PYG{p}{,}
    \PYG{l+s+s2}{\PYGZdq{}}\PYG{l+s+s2}{token}\PYG{l+s+s2}{\PYGZdq{}}\PYG{p}{:} \PYG{l+s+s2}{\PYGZdq{}}\PYG{l+s+s2}{T6v23Tnw95S8BX2yNR8Os8RB6L4HAnvhTTrfjmxB5UyMaef}\PYG{l+s+s2}{\PYGZdq{}}
\PYG{p}{\PYGZcb{}}
\end{sphinxVerbatim}

Con el token se pueden hacer las peticiones que se deseen al recurso
\sphinxcode{\sphinxupquote{visión}}. Supongamos que queremos analizar una imagen para encontrar rostros, objetos, traducir 
texto y clasificar de escenarios. Se hace como sigue:

\fvset{hllines={, ,}}%
\begin{sphinxVerbatim}[commandchars=\\\{\}]
curl \PYGZhy{}X POST \PYG{l+s+se}{\PYGZbs{}}
http://132.248.180.17/vision \PYG{l+s+se}{\PYGZbs{}}
\PYGZhy{}H \PYG{l+s+s1}{\PYGZsq{}authorization: T6v23Tnw95S8BX2yNR8Os8RB6L4HAnvhTTrfjmxB5UyMaef\PYGZsq{}} \PYG{l+s+se}{\PYGZbs{}}
\PYGZhy{}H \PYG{l+s+s1}{\PYGZsq{}content\PYGZhy{}type: application/json\PYGZsq{}} \PYG{l+s+se}{\PYGZbs{}}
\PYGZhy{}d \PYG{l+s+s1}{\PYGZsq{}\PYGZob{}}
\PYG{l+s+s1}{\PYGZdq{}imageSource\PYGZdq{} : \PYGZdq{}https://www.robotshop.com/blog/en/files/NAO\PYGZhy{}Hanover.jpg\PYGZdq{},}
\PYG{l+s+s1}{   \PYGZdq{}features\PYGZdq{} : [}
\PYG{l+s+s1}{      \PYGZob{}}
\PYG{l+s+s1}{        \PYGZdq{}type\PYGZdq{}: \PYGZdq{}FACE\PYGZus{}RECOGNITION\PYGZdq{}}
\PYG{l+s+s1}{      \PYGZcb{},}
\PYG{l+s+s1}{      \PYGZob{}}
\PYG{l+s+s1}{        \PYGZdq{}type\PYGZdq{} : \PYGZdq{}OBJECT\PYGZus{}DETECTION\PYGZdq{}}
\PYG{l+s+s1}{      \PYGZcb{},}
\PYG{l+s+s1}{      \PYGZob{}}
\PYG{l+s+s1}{        \PYGZdq{}type\PYGZdq{} : \PYGZdq{}OCR\PYGZus{}TRANSLATION\PYGZdq{}}
\PYG{l+s+s1}{      \PYGZcb{},}
\PYG{l+s+s1}{      \PYGZob{}}
\PYG{l+s+s1}{        \PYGZdq{}type\PYGZdq{} : \PYGZdq{}CLASSIFY\PYGZus{}INDOOR\PYGZus{}SCENES\PYGZdq{}}
\PYG{l+s+s1}{      \PYGZcb{}}\PYG{l+s+s1}{]}\PYG{l+s+s1}{\PYGZcb{}\PYGZsq{}}
\end{sphinxVerbatim}

La API de CloudNAO envía el el cuerpo de su respuesta el
siguiente JSON:

\fvset{hllines={, ,}}%
\begin{sphinxVerbatim}[commandchars=\\\{\}]
\PYG{p}{\PYGZob{}}
    \PYG{l+s+s2}{\PYGZdq{}features\PYGZdq{}}\PYG{o}{:} \PYG{p}{\PYGZob{}}
        \PYG{l+s+s2}{\PYGZdq{}objectDetection\PYGZdq{}}\PYG{o}{:} \PYG{p}{[}
            \PYG{p}{\PYGZob{}}
                \PYG{l+s+s2}{\PYGZdq{}confidence\PYGZdq{}}\PYG{o}{:} \PYG{l+m+mf}{0.9916030764579773}\PYG{p}{,}
                \PYG{l+s+s2}{\PYGZdq{}category\PYGZdq{}}\PYG{o}{:} \PYG{l+s+s2}{\PYGZdq{}person\PYGZdq{}}\PYG{p}{,}
                \PYG{l+s+s2}{\PYGZdq{}topLeftY\PYGZdq{}}\PYG{o}{:} \PYG{l+m+mi}{42}\PYG{p}{,}
                \PYG{l+s+s2}{\PYGZdq{}height\PYGZdq{}}\PYG{o}{:} \PYG{l+m+mi}{158}\PYG{p}{,}
                \PYG{l+s+s2}{\PYGZdq{}topLeftX\PYGZdq{}}\PYG{o}{:} \PYG{l+m+mi}{430}\PYG{p}{,}
                \PYG{l+s+s2}{\PYGZdq{}width\PYGZdq{}}\PYG{o}{:} \PYG{l+m+mi}{50}
            \PYG{p}{\PYGZcb{}}\PYG{p}{,}
            \PYG{p}{\PYGZob{}}
                \PYG{l+s+s2}{\PYGZdq{}confidence\PYGZdq{}}\PYG{o}{:} \PYG{l+m+mf}{0.9870478510856628}\PYG{p}{,}
                \PYG{l+s+s2}{\PYGZdq{}category\PYGZdq{}}\PYG{o}{:} \PYG{l+s+s2}{\PYGZdq{}person\PYGZdq{}}\PYG{p}{,}
                \PYG{l+s+s2}{\PYGZdq{}topLeftY\PYGZdq{}}\PYG{o}{:} \PYG{l+m+mi}{79}\PYG{p}{,}
                \PYG{l+s+s2}{\PYGZdq{}height\PYGZdq{}}\PYG{o}{:} \PYG{l+m+mi}{115}\PYG{p}{,}
                \PYG{l+s+s2}{\PYGZdq{}topLeftX\PYGZdq{}}\PYG{o}{:} \PYG{l+m+mi}{253}\PYG{p}{,}
                \PYG{l+s+s2}{\PYGZdq{}width\PYGZdq{}}\PYG{o}{:} \PYG{l+m+mi}{49}
            \PYG{p}{\PYGZcb{}}\PYG{p}{,}
            \PYG{p}{\PYGZob{}}
                \PYG{l+s+s2}{\PYGZdq{}confidence\PYGZdq{}}\PYG{o}{:} \PYG{l+m+mf}{0.9252263307571411}\PYG{p}{,}
                \PYG{l+s+s2}{\PYGZdq{}category\PYGZdq{}}\PYG{o}{:} \PYG{l+s+s2}{\PYGZdq{}person\PYGZdq{}}\PYG{p}{,}
                \PYG{l+s+s2}{\PYGZdq{}topLeftY\PYGZdq{}}\PYG{o}{:} \PYG{l+m+mi}{11}\PYG{p}{,}
                \PYG{l+s+s2}{\PYGZdq{}height\PYGZdq{}}\PYG{o}{:} \PYG{l+m+mi}{148}\PYG{p}{,}
                \PYG{l+s+s2}{\PYGZdq{}topLeftX\PYGZdq{}}\PYG{o}{:} \PYG{l+m+mi}{2}\PYG{p}{,}
                \PYG{l+s+s2}{\PYGZdq{}width\PYGZdq{}}\PYG{o}{:} \PYG{l+m+mi}{53}
            \PYG{p}{\PYGZcb{}}\PYG{p}{,}
            \PYG{p}{\PYGZob{}}
                \PYG{l+s+s2}{\PYGZdq{}confidence\PYGZdq{}}\PYG{o}{:} \PYG{l+m+mf}{0.9083054661750793}\PYG{p}{,}
                \PYG{l+s+s2}{\PYGZdq{}category\PYGZdq{}}\PYG{o}{:} \PYG{l+s+s2}{\PYGZdq{}sports ball\PYGZdq{}}\PYG{p}{,}
                \PYG{l+s+s2}{\PYGZdq{}topLeftY\PYGZdq{}}\PYG{o}{:} \PYG{l+m+mi}{248}\PYG{p}{,}
                \PYG{l+s+s2}{\PYGZdq{}height\PYGZdq{}}\PYG{o}{:} \PYG{l+m+mi}{50}\PYG{p}{,}
                \PYG{l+s+s2}{\PYGZdq{}topLeftX\PYGZdq{}}\PYG{o}{:} \PYG{l+m+mi}{219}\PYG{p}{,}
                \PYG{l+s+s2}{\PYGZdq{}width\PYGZdq{}}\PYG{o}{:} \PYG{l+m+mi}{49}
            \PYG{p}{\PYGZcb{}}\PYG{p}{,}
        \PYG{p}{]}\PYG{p}{,}
        \PYG{l+s+s2}{\PYGZdq{}indoorScenesClassify\PYGZdq{}}\PYG{o}{:} \PYG{p}{\PYGZob{}}
            \PYG{l+s+s2}{\PYGZdq{}indoor\PYGZus{}scene\PYGZdq{}}\PYG{o}{:} \PYG{l+s+s2}{\PYGZdq{}soccer\PYGZus{}court\PYGZdq{}}
        \PYG{p}{\PYGZcb{}}
    \PYG{p}{\PYGZcb{}}\PYG{p}{,}
    \PYG{l+s+s2}{\PYGZdq{}errors\PYGZdq{}}\PYG{o}{:} \PYG{p}{\PYGZob{}}
        \PYG{l+s+s2}{\PYGZdq{}faceRecognition\PYGZdq{}}\PYG{o}{:} \PYG{p}{\PYGZob{}}
            \PYG{l+s+s2}{\PYGZdq{}message\PYGZdq{}}\PYG{o}{:} \PYG{l+s+s2}{\PYGZdq{}invalid url was sent\PYGZdq{}}
        \PYG{p}{\PYGZcb{}}\PYG{p}{,}\PYG{l+s+s2}{\PYGZdq{}ocrTranslation\PYGZdq{}}\PYG{o}{:} \PYG{p}{\PYGZob{}}
            \PYG{l+s+s2}{\PYGZdq{}message\PYGZdq{}}\PYG{o}{:} \PYG{l+s+s2}{\PYGZdq{}Text not found\PYGZdq{}}
        \PYG{p}{\PYGZcb{}}
    \PYG{p}{\PYGZcb{}}\PYG{p}{\PYGZcb{}}
\end{sphinxVerbatim}




