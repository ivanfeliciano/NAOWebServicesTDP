
\section{Servicios Web REST}
\label{\detokenize{chapter_one/rest:arquitecturas-de-servicios-de-transferencia-de-estado-representacional}}\label{\detokenize{chapter_one/rest::doc}}


\subsection{Transferencia de Estado Representacional (REST)}
\label{\detokenize{chapter_one/rest:transferencia-de-estado-representacional-rest}}
% En el año 2000, después de que la crisis de la escalabilidad de la web se
% evitó, Fielding llamó y describió al estilo arquitectónico de la web en su
% trabajo doctoral. \sphinxstylestrong{Transferencia de Estado Representacional} (REST por
% sus siglas en inglés) fue el nombre que Fielding dio a su descripción del
% estilo arquitectónico de la web, que está compuesto por las restricciones
% mencionadas previamente.

La Transferencia de Estado Representacional es un estilo de arquitectura de software.
Este estilo es
una abstracción de elementos arquitectónicos dentro de un sistema de hipermedia
distribuido como es la Web. REST ignora los detalles de la implementación de
componentes y sintaxis de protocolos de manera que pueda enfocarse en los
papeles de los componentes, las restricciones sobre su interacción con otros
componentes, y la interpretación de elementos de datos significativos.
Abarca las limitaciones fundamentales sobre los componentes, conectores y datos
que definen las bases de la arquitectura web y, por lo tanto, la esencia de su
comportamiento como una aplicación basada en red. REST no es un estándar, sin
embargo sí un conjunto de restricciones. No está atado al protocolo HTTP,
pero a menudo se asocia con éste.


\subsubsection{API REST}
\label{\detokenize{chapter_one/rest:api-rest}}
Un \sphinxstyleemphasis{servicio web} es un sistema de software diseñado para admitir
la interacción interoperable de una máquina a otra  máquina a través de una red.
Programas cliente usan \sphinxstyleemphasis{interfaces de programación de aplicaciones} (API
por sus siglas en inglés) para comunicarse con servicios web. Una API expone un conjunto de datos y funciones para facilitar
interacciones entre programas de computadora y permitiendo que
intercambien información.

\begin{figure}[htbp]
\centering


\noindent\sphinxincludegraphics[scale=0.60]{{web_service_cycle}.png}
\caption{Una API web es el frente de un servicio web, escuchando y respondiendo las peticiones de los cliente directamente.}\label{\detokenize{chapter_one/rest:web-service-cycle}}\end{figure}

El estilo arquitectónico REST se aplica comúnmente al diseño de API para
servicios web modernos. Una API web que sigue el estilo REST es una API REST.
Tener una API REST hace a un servicio web RESTful. Una API REST está formada de
recursos entrelazados.\\

REST es una arquitectura basada en recursos. Se accede a un recurso a través
de una interfaz común basada en los métodos estándar de HTTP. REST
solicita a los desarrolladores usar métodos HTTP explícitamente
y de una forma que sea consistente con la definición del protocolo.
Cada recurso se identifica con un URL. Todos los recursos deben soportar
las operaciones HTTP más comunes, además REST permite que ese recurso
tenga diferentes representaciones, por ejemplo, texto, XML, JSON, etc.
El cliente REST puede solicitar una representación en específico por
medio del protocolo HTTP. El cuadro \ref{\detokenize{chapter_one/rest:rest-struct}} describe
los elementos usados en REST.



\begin{table}
\centering
\caption{Elementos de REST}\label{\detokenize{chapter_one/rest:rest-struct}}
\begin{tabulary}{\linewidth}[t]{|T|T|}
\hline
\sphinxstylethead{ 
Elemento
\unskip}\relax &\sphinxstylethead{ 
Descripción
\unskip}\relax \\
\hline
Recurso
&
Objetivo conceptual de una referencia de hipertexto. Por ejemplo: podcast.
\\
\hline
Identificador de recurso
&
Un URL que identifica un recurso en específico. Por ejemplo: \sphinxurl{http://convoynetwork.com/podcast/123}
\\
\hline
Metadatos del recurso
&
Información que describe al recurso. Por ejemplo: autor, etiqueta, etc.
\\
\hline
Representación
&
El contenido del recurso. Por ejemplo: un JSON, un HTML o una imagen JPEG.
\\
\hline
Metadatos de la representación
&
Información que describe como procesar la representación. Por ejemplo: tipo de medio, fecha, etc.
\\
\hline
Datos de control
&
Información que describe cómo optimizar el procesamiento de respuesta. Por ejemplo: if-modified-since, cache-control-expiry.
\\
\hline
\end{tabulary}
\end{table}
