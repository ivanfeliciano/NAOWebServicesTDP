

% Interconnection of sensing and actuating devices providing the ability to share information across platforms through a unified framework, developing a common operating picture for enabling innovative applications. This is achieved by seamless ubiquitous sensing, data analytics and information representation with Cloud computing as the unifying framework.

Actualmente existen diferentes plataformas que 
ofrecen servicios
de cómputo en la nube para solucionar problemas de 
visión computacional \cite{googlevision2018} o
procesamiento de lenguaje natural \cite{witaidocs2018} usando modelos
de aprendizaje profundo. Éstas también incluyen servicios 
para entrenar y lanzar
modelos creados por los mismos usuarios \cite{tensorflowgooglecloud2018}.
Todos estos recursos cuentan con interfaces de programación de aplicaciones
con una arquitectura de transferencia de estado representacional o REST (por sus siglas en inglés, representational state transfer) la cual 
permite que cualesquiera dispositivos que manejen el protocolo HTTP
puedan acceder a éstos.
El cómputo en la nube es un estilo de cómputo que 
permite el acceso a 
recursos informáticos a través de internet cuando un
usuario lo demanda \cite{borkofurhtarmandoescalante2010}. El aprendizaje profundo es un 
tipo
de aprendizaje automático donde las computadoras 
construyen
conceptos complejos a partir de conceptos más 
simples \cite{iangoodfellowyoshuabengioaaroncourville2017}.


%¿Qué es lo que no sabemos?\\
% El cómputo en la nube es el marco de trabajo que
% unifica los elementos que permiten que exista el
% Internet de las cosas \cite{jayavardhanagubbiarajkumarbuyyabslavenmarusicamarimuthupalaniswamia2013}. Queremos que el robot NAO, Fibonacho, 
% sea parte se esa interconexión de sensores y actuadores
% que comparten información entre plataformas
% a través de internet.
Fibonacho es un robot humanoide programable y autónomo
que a pesar de contar con
un poder de procesamiento suficiente para realizar 
tareas como
jugar un partido de fútbol con otros robots \cite{splinfo2018},
las aplicaciones modernas
exigen que los dispositivos se adapten constantemente a nuevas necesidades
de los usuarios, como el acceso a los datos
del robot a cualquier hora y desde cualquier lugar, el uso del aprendizaje profundo para resolver problemas como la 
clasificación de imágenes en múltiples categorías, encontrar
la posición de objetos en una imagen, reconocimiento
de personas y el 
procesamiento de lenguaje natural de un discurso \cite{benkehoesachinpatilpieterabbeelkengoldberg2014}.

%¿Qué pretendemos averiguar? (objetivo)

En este trabajo se integraron servicios en la nube
con el robot Fibonacho.
Los recursos son brindados por las plataformas
Google Cloud, Wit.ai, Kairos y por
el mismo Laboratorio de Algoritmos para la
Robótica.
Estos servicios en la nube permiten al robot
realizar tareas como el procesamiento de lenguaje
natural de un discurso oral, el procesamiento 
de imágenes para reconocimiento de rostros
y objetos y la clasificación de
imágenes en escenarios.